\documentclass[a4paper]{article}

\usepackage[british]{babel}
% Non-indented paragraphs
\usepackage{parskip}
\usepackage[margin=2.54cm,includefoot]{geometry}
\usepackage{color}
\usepackage[colorlinks,pagebackref,pdfusetitle,plainpages=false,pdfstartview=FitH]{hyperref}

\title{Intermediate LaTeX, 24 February 2009}
\author{Anthony Smith\thanks{A.J.Smith@sussex.ac.uk}}

%-------------------------------
\begin{document}
%-------------------------------

\maketitle

%-------------------------------
\section{Introduction}
%-------------------------------

Outcomes:
\begin{itemize}
\item Finish thesis\footnote{Apart from the content}
\item Good working knowledge of bibliographies, figures, tables
\item Know where to go for further information
\end{itemize}

If you want a GUI, you could try...
\begin{itemize}
\item TeXnicCenter (Windows)
\item TeXShop (Mac)
\item Texmaker (Linux, Windows, Mac)
\end{itemize}

Traditional: \LaTeX\ (\TeX) producing DVI file, convert to PS/PDF

But probably better: pdf\LaTeX\ (pdf\TeX), producing PDF file (easy to include hyperlinks)

%-------------------------------
\section{Getting started}
%-------------------------------

Each document begins with \verb=\documentclass{...}=, e.g., \verb=article=. Journals often provide their own class files (\verb=.cls=). We're going to use a Sussex thesis class, based on the \verb=report= class.

The instructions below assume you are using TeXnicCenter on the IT Services PCs. An alternative would be to log in to a remote (probably Linux) machine by starting Exceed (Start --- All Programs --- Applications --- Hummingbird Connectivity --- Exceed) and then logging in using PuTTy (Start --- All Programs --- Utilities --- PuTTy). 

\begin{enumerate}
  \item Create a new folder for your latex files.
  \item Download the \verb=thesis.zip= file from \verb=<http://astronomy.sussex.ac.uk/~anthonys/latex/usthesis/>= to this folder, and extract the contents.
  \item Start --- All Programs --- Applications --- TeXnicCenter 1 --- TeXnicCenter
  \item Open \verb=thesis.tex= and \verb=bib.bib=. Have a look!
  \item With \verb=thesis.tex= selected, choose ``Project" --- ``Create with active file as main file", selecting ``Use BibTeX", and Project language ``en" (in case that actually makes a difference).
  \item Select ``Build --- Build and view output" (Ctrl+F5) to compile and view. Do it again and it should work; it needs to be done twice to get all the cross-references correct. (When using latex ``manually", the sequence of commands is: pdflatex thesis---bibtex thesis---pdflatex thesis---pdflatex thesis, in order to get the bibliography and cross-references right.)
\end{enumerate}

%-------------------------------
\section{Managing large documents}
%-------------------------------

\begin{itemize}
\item On the left-hand side of the TeXnicCenter window, you will notice the structure of the thesis.tex file, with chapters and sections listed.  Try double-clicking on the headings in the structure pane to navigate to different parts of the document.
\item You will probably find it easier to have each chapter as a separate file. Try this as follows:
\begin{enumerate}
\item ``File --- New..." and create a new file for a chapter between the introduction and the conclusions.  E.g., \verb=methods.tex=
\item In this new file, type \verb=\chapter{Methods of analysing the data}= or whatever you want the chapter to be called.  Add some text a couple of lines below as the first sentence of the chapter.
\item Now return to thesis.tex, find the position just above the Conclusions, and enter \verb=\input{methods}= (without \verb=.tex=).
\item Now build the document again (you will need to do it twice), and your new chapter should appear, both in the output PDF file and in the structure pane in TeXnicCenter.
\item Double click on the chapter heading in the TeXnicCenter structure pane to navigate to the chapter
\end{enumerate}
\end{itemize}

%-------------------------------
\section{Bibliographies}
%-------------------------------

Essentials:
\begin{itemize}
  \item Bibliographic information is stored in one or more databases, \verb=*.bib=. Take a look inside \verb=bib.bib=.
  \item The bibliography is inserted using \verb=\bibliography{=\emph{database(s)}\verb=}= (omit \verb=.bib= from the filenames).
  \item Refer to the items using \verb=\cite=, or (with package \verb=natbib=) \verb=\citet= or \verb=\citep= for (t)extual or (p)aranthetical citations respectively. Text can be added inside the parantheses; for example, \verb=\citep[e.g., ][pg 22]{S07}= might produce: (e.g., Smith, 2007, pg 22).
  \item \verb=\bibliographystyle{...}= needs to be specified at some point (e.g., the preamble). There are many bibliography style (\verb=.bst=) files; journals often have their own. For non--author-year citation, you could try: \verb=plain=, \verb=unsrt=, \verb=abbrv=, \verb=siam=, \verb=ieeetr= or \verb=alpha=. For author-year, you could try: \verb=plainnat=, \verb=unsrtnat=, \verb=abbrvnat= or \verb=apalike=. But there are more, and you can make your own. I'm currently using one like \verb=apalike=.
  \item The \verb=natbib= package is recommended. It defaults to author-year citations, but with \verb=[numbers]= or \verb=[super]= it gives numerical or superscripted-numerical citations respectively. \verb=[round]= gives round parantheses (may be default).
\end{itemize}

It is easiest to get BibTeX data from an online source: see course website. Or, if necessary, you can make your own entries: see templates in \verb=bib.bib=.

Exercise:
\begin{enumerate}
\item Find and download BibTeX data for two or three papers.
\item For another (fictional?) paper or book, create your own BibTeX entry using a template.
\item Cite these sources in the thesis file, using \verb=\citep=, and \verb=\citet=.
\item Compile using various bibliography styles (NB latex---bibtex---latex---latex).
\end{enumerate}



%-------------------------------
\section{Figures}
%-------------------------------

The \verb=graphicx= package is marginally easier to use than \verb=graphics=, so we'll use that. It \emph{may} be necessary to include \verb=[pdftex]= as option for the \verb=graphicx= package.

With pdf\LaTeX, many image formats are accepted.  However, EPS files will need to be converted, but this can be done using the \verb=epstopdf= package, which automatically converts the files to PDF. Simply include that in the preamble. (I couldn't get this to work with TeXnicCenter, but it certainly works on other systems.)

Don't include the extension in the filename; it will work it out. E.g., for \verb=image.jpg= use \verb=\includegraphics= \verb={image}=. But make sure the filenames are unique!

Exercise:
\begin{enumerate}
  \item Download image files from course website
  \item Include them all as figures, with captions (above or below) and labels.
  \item Run latex a couple of times; the figures should appear in the List of Figures.
  \item Experiment with size/rotation using options in \verb+\includegraphics[+\emph{options}\verb+]{+\emph{filename}\verb+}+. E.g., \verb+width=+ (length, such as \verb=5cm= or \verb=0.8\linewidth= or \verb=\columnwidth=), \verb+height=+ (if height and width are not both specified, scaling is preserved), \verb+angle=+ in degrees (use \emph{before} or \emph{after} height and width, depending on which you want to do first).
  \item Experiment with positioning using \verb=\centering= and/or using \verb=\begin{figure}[...]= where \verb=...= is some combination of h(ere), t(op), b(ottom) or p(age). If no option is given, \LaTeX assumes \verb=[tbp]=. (This is the same as for tables.)
  \item \verb=\suppressfloats= prevents any further floats (figures or tables) from appearing on the current page. \verb=\suppressfloats[t]= prevents any more appearing at the top of the page (etc).
  \item Try putting two figures next to each other using \verb=\begin{figure}= then \verb=\begin{minipage}[t]{5.0cm}= then the left-hand figure and caption, then \verb=\end{minipage} \hfill= followed by the second \verb=minipage= and \verb=\end{figure}=.
\end{enumerate}

%-------------------------------
\section{Tables}
%-------------------------------

There is a \verb=tabbing= environment for simple tab stops within the text. It's quite straightforward. However, for larger tables, the \verb=tabular= or \verb=array= environments are more suitable. If you want to float the table (with a number and a caption), the table must be placed within a \verb=table= environment.

The \verb=array= environment is only available in mathematical mode.

Exercise:
\begin{enumerate}
  \item Fetch table data from course website (it's not very exciting) and put it somewhere in the thesis.
  \item Place it within a \verb=\begin{tabular}{ccccccc}= environment, which should go within a \verb=table= environment. The seven c's indicate seven columns, all centred.
  \item Add a \verb=\centering= just before the \verb=tabular= environment
  \item Add a caption at the top (with a label and a shortened form for the contents).
  \item Run latex a couple of times; the table should appear in the List of Tables.
  \item Experiment with (l)eft-, (c)entre- and (r)ight-aligned columns in \verb=\begin{tabular}=.
  \item Try putting some vertical lines in, e.g., \verb={c||cc|c|c|cc}=.
  \item Try some horizontal lines, \verb=\hline=, between the rows or at the top of the table. You could put two \verb=\hline= commands together for two lines.
  \item Problem: the table it too wide. Try these three solutions:
	\begin{enumerate}
	    \item Use \verb={cccp{2cm}ccc}=
	    \item Use \verb=\parbox[t][1.2\totalheight]{2cm}{This is a very long column name indeed}= 
	    
	    \parbox[t][1.2\totalheight]{2cm}{This is a very long column name indeed}
	    \item Place the whole table landscape on a page by itself. Place \verb=\usepackage{lscape}= in the preamble, then \verb=\begin{landscape} ... \end{landscape}= around the table.
	\end{enumerate}
  \item Math mode tables: 
  \begin{enumerate}
  \item replace \verb=tabular= with \verb=array=
  \item Place the \verb=array= environment within \verb=\[= and \verb=\]= (math mode)
  \item Remove the column headings line
  \item Now experiment with adding mathematical notation (variable names, powers, \verb=\sin^{-2}{\theta)= etc)
  \end{enumerate}
\end{enumerate}

%-------------------------------
\section{Little exercises}
%-------------------------------

\begin{enumerate}
\item Make a short version of your favourite \LaTeX\ command, to save typing it out in full every time you use it: add \verb=\newcommand{\abc}{blah blah}= to the preamble to insert \verb=blah blah= whenever you type \verb=\abc= (but do it for something more useful!).
\item Try some cross-referencing: use \verb=\label{fig:xyz}=, then refer to them with \verb=Chapter \ref{chap:res}= and \verb=Page \pageref{chap:res}=, \emph{not} the numbers, which change!
\item Try keeping chapters as separate files: use \verb=\input{results}= for \verb=results.tex=
\item Put some text in colour: \verb=\usepackage{color}=  (Package may require \verb=[pdftex]= option.)
\begin{itemize}
\item \textcolor{blue}{this is blue} \verb=\textcolor{blue}{this is blue}= 
\item \verb=\color{blue}= \color{blue} to set current colour to blue \color{black}
\item \colorbox{yellow}{Yellow background} \verb=\colorbox{yellow}{Yellow background}=
\end{itemize}
\item Add some hyperlinks (within footnotes?), which work within the PDF output: \verb=hyperref= package (see \verb=thesis.tex=).
\begin{itemize}
\item \verb=\url{http://www.google.co.uk}= for \url{http://www.google.co.uk}
\item \verb=\href{http://www.google.co.uk}{Google}= for \href{http://www.google.co.uk}{Google}
\item \verb=\href{http://www.google.co.uk}{\nolinkurl{google.co.uk}}= for \href{http://www.google.co.uk}{\nolinkurl{google.co.uk}}
\end{itemize}
\end{enumerate}

%-------------------------------
\section{Miscellany}
%-------------------------------

\begin{itemize}
\item \verb=\usepackage{amsmath}= for decent mathematical output. Use \verb=\begin{align}= (or \verb=align*= for unnumbered equations) rather than \verb=eqnarray=, and use \verb=\[...\]= rather than \verb=\begin{displaymath}= or \verb=$$...$$=. Or use \verb=\begin{equation}= for numbered equations.
\item Customising headers and footers: \verb=\usepackage{fancyhdr}=
\item Adjusting margins: \verb=\usepackage{geometry}=
\item Diagrams: use \verb=picture= environment.
\end{itemize}

\end{document}
